\newcommand{\blankpage}{\null\thispagestyle{empty}\addtocounter{page}{-1}\newpage}
\newcommand{\emptypage}{\clearpage}

\begin{titlepage}
    \includegraphics[width=36mm]{../../source/_static/img/logo-cafoscari.jpg}\\[20mm] % Logo

    \begin{center}
        \Large Bachelor's Degree\\in Informatics\\[20mm]
        \Large Final Thesis\\[10mm]
        \textbf{\Huge A study on\\Equalization Curve Detection\\in Audio Tape Digitization process\\using Artificial Intelligence}\\[20mm]
    \end{center}

    \textbf{Supervisor}\\Ch. Prof. Ilaria Prosdocimi\\[10mm]
    \textbf{Graduand}\\Matteo Spanio\\Matricolation number\\877485\\[10mm]
    \textbf{Academic Year}\\2021/2022

\end{titlepage}

\blankpage
\onehalfspacing
\pagenumbering{roman}

\chapter*{Acknowledgements}

I would like to take this opportunity to express my gratitude to all those who have contributed to the successful completion of my thesis in computer science.

First and foremost, I would like to thank my family for their constant support and encouragement throughout my academic journey. Their unwavering belief in me has been a source of motivation and inspiration that has kept me going during the challenging times.

I am also deeply grateful to the research group of the CSC laboratory for their warm welcome and support. In particular, I would like to thank Nadir Dalla Pozza for his guidance and assistance in navigating the intricate world of MPAI. Without his help, I would have been lost and unable to overcome the obstacles that I encountered.

I would also like to express my gratitude to Niccolò Pretto for his supervision of my work. His guidance and expertise have been invaluable throughout the entire research process.

Finally, I would like to extend a special thank you to Andrea Pietracaprina and Sergio Canazza, who first introduced me to the CSC laboratory. Without their help, I would not have had the opportunity to work alongside such talented and dedicated researchers.

Once again, I extend my sincere thanks to all those who have supported me on this journey. Your contributions have been invaluable, and I will always be grateful for your help and encouragement.

\null
\thispagestyle{empty}
\newpage

\chapter*{Abstract}

In recent decades, archives have seen a rapid change in the media used to store sound information, and many of these media are rich in obsolete material that risks becoming unusable due to aging. Therefore, it is necessary to digitize sound documents in order to make them durable over time. However, during the digitization process, errors such as applying an incorrect equalization curve or playing back the tape at the wrong speed can lead to the acquisition of inauthentic material.

This work focuses on studying the detection of possible errors due to incorrect equalization curve settings and tape playback speed during the transfer of material from analog to digital, verifying if and how it is possible to detect them using methods specific to Artificial Intelligence (clustering and classification).

The results of this research demonstrate that these algorithms may offer good precision in detecting errors and have the potential to automate the verification process, ensuring the preservation of valid information for a longer period of time, but before they can be used in a real-world scenario, they must be further improved.

\null
\thispagestyle{empty}
\newpage

\blankpage
